\documentclass{article}
\usepackage[a4paper,left=2.5cm,right=2.5cm,top=2.5cm,bottom=2.5cm]{geometry}

\title{Session 5}
\author{Zohaad Fazal}
\date{February 2020}
\usepackage{amsmath}
\DeclareMathOperator*{\argmax}{arg\,max}
\usepackage{amssymb}
\usepackage{bbm}
\usepackage{tikz}
\usepackage{tkz-euclide}
\usetkzobj{all}
\usepackage{cancel}
\usepackage{caption}
\usepackage{textcomp}
\usepackage{eurosym}

% derivative symbol
\renewcommand{\d}[1]{\ensuremath{\operatorname{d}\!{#1}}}

\usetikzlibrary{calc,intersections,through,backgrounds,matrix,patterns}
\usepackage{pgfplots}
\usepgfplotslibrary{fillbetween}
\pgfdeclarelayer{bg}
\pgfsetlayers{bg,main}

\tikzset{
    hatch distance/.store in=\hatchdistance,
    hatch distance=10pt,
    hatch thickness/.store in=\hatchthickness,
    hatch thickness=0.3pt
}

\makeatletter
\pgfdeclarepatternformonly[\hatchdistance,\hatchthickness]{northeast}
{\pgfqpoint{0pt}{0pt}}
{\pgfqpoint{\hatchdistance}{\hatchdistance}}
{\pgfpoint{\hatchdistance-1pt}{\hatchdistance-1pt}}%
{
    \pgfsetcolor{\tikz@pattern@color}
    \pgfsetlinewidth{\hatchthickness}
    \pgfpathmoveto{\pgfqpoint{0pt}{0pt}}
    \pgfpathlineto{\pgfqpoint{\hatchdistance}{\hatchdistance}}
    \pgfusepath{stroke}
}

\pgfdeclarepatternformonly[\hatchdistance,\hatchthickness]{northwest}
{\pgfqpoint{0pt}{0pt}}
{\pgfqpoint{\hatchdistance}{\hatchdistance}}
{\pgfpoint{\hatchdistance-1pt}{\hatchdistance-1pt}}%
{
    \pgfsetcolor{\tikz@pattern@color}
    \pgfsetlinewidth{\hatchthickness}
    \pgfpathmoveto{\pgfqpoint{\hatchdistance}{0pt}}
    \pgfpathlineto{\pgfqpoint{0pt}{\hatchdistance}}
    \pgfusepath{stroke}
}



% derivative symbol
\renewcommand{\d}[1]{\ensuremath{\operatorname{d}\!{#1}}}

\newcommand{\notimplies}{%
  \mathrel{{\ooalign{\hidewidth$\not\phantom{=}$\hidewidth\cr$\implies$}}}}

\begin{document}
\maketitle
\section*{0\quad Topics}
\begin{enumerate}
    \item VCG (Vickrey-Clarke-Groves) Mechanism
    \item Competitive Equilibrium and the Welfare Theorem
\end{enumerate}

\section*{1\quad VCG Mechanism}
\subsection*{Economics}
$N=\{1,\dots,n\}$: set of agents\\
$A$: set of allocations
\begin{equation*}
    v_i:A\to\mathbb{R}\quad\text{for all }i
\end{equation*}
For $a\in A$, $v_i(a)$ is the valuation agent $i$ assigns to allocation $a$.\\

\noindent
Assume quasi-linear utility.\\
\noindent
\textbf{Task:} Choose $a^*\in A$ such that
\begin{equation*}
    \sum_{i\in N}v_i(a^*)\geq\sum_{i\in N}v_i(a)\quad\text{for all }a\in A.
\end{equation*}
This is TU maximization.\\

\noindent
Assume private information: each agent knows only their own valuation, auctioneer knows none of the valuations.\\

\noindent
\textbf{Question:} Can we design or construct a
\begin{itemize}
    \item market
    \item auction
    \item trading platform
\end{itemize}
such that, \textit{in equilibrium}, allocation is efficient? (TU is maximized)\\

\noindent
\textbf{Answer:} yes.
\subsection*{VCG}
\subsubsection*{(1)\quad Strategies}
One shot $\equiv$ sealed bid.\\

\noindent
Each agent $i$ submits report $r_i:A\to\mathbb{R}$. (reported valuation)
\subsubsection*{(2)\quad Allocation}
Choose allocation $b^*\in A$, such that
\begin{equation*}
    \sum_{i\in N}r_i(b^*)\geq\sum_{i\in N}r_i(b)\quad\text{for all }b\in A.
\end{equation*}
This maximizes total reported valuation.
\subsubsection*{(3)\quad Payments (with Externality)}
Externality: damage you inflict by participating; like throwing a party at 3AM and waking the whole neighborhood.\\

\noindent
Agent $i$ is going to pay
\begin{equation*}
    p_i=\underbrace{\sum_{j\neq i}r_j(b^{**})}_{\text{TU if you're not there}}-\underbrace{\sum_{j\neq i}r_j(b^*)}_{\text{what the others currently get}}
\end{equation*}

where
\begin{equation*}
    b^{**}=\argmax_{b\in A}\sum_{j\neq i}r_j(b).
\end{equation*}

\subsection*{Theorem}
In the VCG mechanism, reporting truthfully, so $r_i=v_i$, is a dominant strategy. In the resulting DSE, allocation is efficient.\\

\noindent
\textit{Proof:} 2\textsuperscript{nd} statement is obvious, only prove the 1\textsuperscript{st} statement.\\

\noindent
Consider agent $i$, other agents' reports $(r_j)_{j\neq i}$. Take report $r_i$ of agent $i$ knowing $v_i$.
\begin{equation*}
    \begin{aligned}
    u_i(b^*)&=v_i(b^*)-p_i\\
    &=v_i(b^*)-\left[\sum_{j\neq i}r_j(b^{**})-\sum_{j\neq i}r_j(b^*)\right]\\
    &=\underbrace{v_i(b^*)+\sum_{j\neq i}r_j(b^*)}_{b^* \text{ depends on }r_i}-\underbrace{\sum_{j\neq i}r_j(b^{**})}_{\text{does not depend on }r_i}
    \end{aligned}
\end{equation*}\\

\noindent
The mechanism is going to pick a $b^*$ such that $r_i(b^*)+\sum_{j\neq i}r_j(b^*)$ is maximal.\\

\noindent
Agent $i$ wants to maximize $v_i(b^*)+\sum_{j\neq i}r_j(b^*)$, this is done via choosing $r_i=v_i$. \ensuremath{\square}

\subsection*{Usage in practice}
This is not used in real life, why? Suppose there are 10 bidders and 100 regional licenses (e.g. mobile telephone companies bidding for regional network licenses).\\

\noindent
There are $10^{100}$ possible allocations, so in VCG every company has to report $10^{100}$ valuations. This is practically impossible.

\subsection*{Example}
5 identical items\\
9 agents\\
unit demand: there is only a valuation for 1 item, marginal valuation for a 2\textsuperscript{nd} item is 0.

\begin{equation*}
    \begin{aligned}
     &v_1=15&*\\
    **\quad &v_2=7&\\
    **\quad &v_3=10&*\\
    **\quad &v_4=12&*\\
    &v_5=5&\\
    &v_6=6&\\
    &v_7=3&\\
    **\quad&v_8=13&*\\
    **\quad&v_9=8&*
    \end{aligned}
\end{equation*}\\

\noindent
$p_1=13+12+10+8+7-(13+12+10+8)=7$, pay the highest excluded bid. Similarly, $p_3=p_4=p_8=p_9=7$.\\

\noindent
Note: with 1 item this is equivalent to the Vickrey auction.

\section*{2\quad Competitive Equilibrium and the Welfare Theorem}
Exchange economy: trade without production\\
Consider economy with trade and production.

\noindent
2 consumers, 1 firm\\
2 goods: wheat and beer\\

\noindent
Firm 1 produces beer from wheat: $f(w)=\sqrt{w}$\\

\noindent
Consumer 1 has utility $u(w,b)=w\cdot b$\\
Initial endowment: None, owns firm\\

\noindent
Consumer 2 has utility $v(w,b)=w\cdot b^2$\\
Initial endowment: owns 2 units of wheat\\

\noindent
Prices: 1 for wheat, $q$ for beer.\\

\noindent
Three equilibrium concepts that are the same:
\begin{itemize}
    \item competitive equilibrium
    \item Walrasian equilibrium
    \item price equilibrium
\end{itemize}

\noindent
Assume a price-taker model. We are going to check whether these prices are equilibrium prices.\\

\noindent
Currently both consumers have 0 utility, so they both have incentive to start trading.\\

\noindent
\textbf{C2} sells his wheat and buys beer with that money. Assume he sells all of his wheat. His budget is now 2.

\begin{equation*}
    \begin{aligned}
        \max\quad &v(w,b)=w\cdot b^2\\
        \text{subject to:}\quad&w + q\cdot b = 2\quad\text{(non-satiation, so equality)}
    \end{aligned}
\end{equation*}\\

\noindent
Using Lagrange: $\mathcal{L}=w\cdot b^2-\lambda(w+q\cdot b-2)$

\begin{equation*}
    \left.
    \begin{aligned}
        \frac{\partial \mathcal{L}}{\partial w}&=b^2-\lambda&=0\\ \\
        \frac{\partial \mathcal{L}}{\partial b}&=2wb-\lambda q&=0
    \end{aligned}
    \quad
    \right]
    \begin{aligned}
        &\times q\\
        &\\
        &\\
        &\\
    \end{aligned}
    \quad\quad\implies
    \begin{aligned}
    &\\
    q\cdot b^2&=2wb\\
    q\cdot b &= 2w
    &\\
    \end{aligned}
\end{equation*}
Using the constraint $w+q\cdot b=2$ we get for consumer 2 $b=\frac{4}{3q}$ and $w=\frac{2}{3}$. This is his demand function.\\

\noindent
\textbf{C1} has only technology, no initial endowments. He operates the firm, wants to maximize profit
\begin{equation*}
    \begin{aligned}
    \max\quad&\Pi_f=q\cdot b-w\\
    \text{subject to:}\quad&b=\sqrt{w}\quad\text{(non-satiation)}
    \end{aligned}
\end{equation*}
By substitution,
\begin{equation*}
    \max\quad q\cdot\sqrt{w}-w
\end{equation*}
First order condition:
\begin{equation*}
    \frac{\d \Pi_f}{\d w}=-\frac{q}{2\sqrt{w}}-1=0\iff w=\frac{q^2}{4}
\end{equation*}
Then, $b=\sqrt{\frac{q^2}{4}}=\frac{q}{2}$. The resulting profit is then $\Pi_f=q\cdot b - w = \frac{q^2}{2}-\frac{q^2}{4}=\frac{q^2}{4}$. This is the budget consumer 1 goes to market with.\\

\noindent
He wishes to maximize utility
\begin{equation*}
    \begin{aligned}
        \max\quad&u(w,b)=w\cdot b\\
        \text{subject to:}\quad&w+q\cdot b = \frac{q^2}{4}
    \end{aligned}
\end{equation*}
This is equivalent to
\begin{equation*}
    \max_b\quad\left(\frac{q^2}{4}-q\cdot b\right)\cdot b
\end{equation*}
The maximum is halfway between the roots, so $b=\frac{1}{2}\cdot\frac{q}{4}=\frac{q}{8}$. Then $w=\frac{q^2}{4}-\frac{q^2}{8}=\frac{q^2}{8}$.

\subsection*{Competitive Equilibrium}
$(1,q)$ is a competitive equilibrium if the market clears, i.e. total supply must equal to total demand for all goods.
\begin{equation*}
\begin{aligned}
    TS_w&=TD_w\\
    TS_b&=TD_b
\end{aligned}
\end{equation*}
For beer:
\begin{equation*}
    \begin{aligned}
        TS_b&=TD_b\\
        \frac{q}{2}&=\frac{q}{8}+\frac{4}{3q}\\
        \frac{3q^2}{8}&=\frac{4}{3}\\
        q^2&=\frac{32}{9}\\
        q&=\frac{4}{3}\cdot\sqrt{2}
    \end{aligned}
\end{equation*}
For wheat:
\begin{equation*}
    \begin{aligned}
        TS_w&=TD_w\\
        2&=\frac{2}{3}+\frac{q^2}{8}+\frac{q^2}{4}\\
        q&=\frac{4}{3}\cdot\sqrt{2}
    \end{aligned}
\end{equation*}
The market clears for this $q$. We get the same answer for $q$ for the market clearing condition with both beer and wheat.
\subsection*{Welfare Theorem}
In the competitive equilibrium, the outcomes (utilities) are Pareto efficient.

\begin{equation*}
    \begin{aligned}
        \text{C1, in equilibrium:}\quad& w=\frac{4}{9},\,&b=\frac{\sqrt{2}}{6}&\text{ gives }u=\frac{4}{9}\cdot\frac{\sqrt{2}}{6}=\frac{2}{27}\sqrt{2}\\
        \text{C2, in equilibrium:}\quad& w=\frac{2}{3},\,&b=\frac{1}{\sqrt{2}}&\text{ gives }v=\frac{2}{3}\cdot\left(\frac{1}{\sqrt{2}}\right)^2=\frac{1}{3}
    \end{aligned}
\end{equation*}

\subsection*{Note}
This is completely different from the TU maximizer, if we wanted to do that it would look like
\begin{equation*}
    \begin{aligned}
        \max\quad&w_1\cdot b_1+w_2\cdot b_2^2\\
        \text{such that:}\quad&b_1+b_2=b_f\\
        &b_f=\sqrt{w_f}\\
        &w_1+w_2+w_f=2
    \end{aligned}
\end{equation*}
There are no prices, only the distribution is a problem. This is "communism".
\end{document}
